\documentclass{resume} % Usar el estilo personalizado resume.cls

\usepackage[sfdefault]{roboto}
\usepackage[left=0.3in,top=0.3in,right=0.4in,bottom=0.5in]{geometry} % Márgenes del documento
\usepackage[none]{hyphenat}

\usepackage{hyperref}
\hypersetup{
	colorlinks=true,
	linkcolor=blue,
	filecolor=magenta,      
	urlcolor=blue,
}

\name{Getulio Valentin Sanchez} % Tu nombre
\address{\href{mailto:valentin.sanchez.py@gmail.com}{valentin.sanchez.py@gmail.com} \\
	\href{https://github.com/gvso}{GitHub}: gvso \\
	\href{https://www.linkedin.com/in/gvso1/}{LinkedIn}: gvso1 \\
	+1 484 477-3869} % Tu dirección

\begin{document}
	
	%----------------------------------------------------------------------------------------
	%	SECCIÓN DE EXPERIENCIA LABORAL
	%----------------------------------------------------------------------------------------
	
	\begin{rSection}{Experiencia Laboral}
		
		\begin{rSubsection}{Ingeniero de Software}{Lev}{}{Abr 2022 - Presente}
			\item Desarrollé un algoritmo para filtrar y rankear prestamistas basado en criterios para financiación, hojas de condiciones, transacciones similares, velocidad en responder, entre otros criterios.
			\item Construí un servicio que procesa todas las comunicaciones por correo electrónico entre nuestros clientes y usuarios prestamistas.
			\item Desarrollé herramientas y servicios para permitir a los equipos de negocio convertir comunicaciones por correo electrónico no estructuradas en datos estructurados para actualizar automáticamente los estados de los acuerdos y los requisitos de financiación de los prestamistas.
			\item Creé un sistema que comprende las tareas pendientes existentes de los usuarios, analiza las comunicaciones entrantes por correo electrónico y crea o actualiza automáticamente tareas para mantener a los usuarios al día con la financiación de sus acuerdos.en su búsqueda de financiación
		\end{rSubsection}
		
		
		\begin{rSubsection}{Ingeniero de Software}{Google}{}{Jun 2021 - Abr 2022}
			\item Trabajé en redes de alto rendimiento e intercomunicación entre procesos.
			\item Reduje el registro de memoria en un 85\% en nuestra pila de software RDMA (Acceso Directo a Memoria Remota).
			\item Moví flujos de trabajo a dispositivos NIC (Tarjeta/Controlador de Interfaz de Red) para ahorrar hasta un 70\% del uso de CPU de nuestro software ejercían sobre los servidores.
			\item Implementé en producción utilizando feature flags y automatizando actualizaciones/degradaciones sin interrupciones.
		\end{rSubsection}
		
		\begin{rSubsection}{Pasante de Ingeniería de Software}{Google}{}{Jun 2020 - Ago 2020}
			\item Reduje los costos operativos de los desarrolladores construyendo una herramienta para automatizar los despliegues graduales de nuevas versiones de código para aplicaciones que se ejecutan en Google Cloud Run.
			\item Implementé y obtuve información concurrentemente sobre varios servicios de Cloud Run a través de la API de Cloud Run utilizando Go.
			\item Determiné automáticamente si las nuevas versiones del código eran saludables basándome en señales de monitoreo de la API de Cloud Monitoring (Stackdriver).
		\end{rSubsection}
		
		\begin{rSubsection}{Pasante de Ingeniería de Software}{Zivtech}{Filadelfia, PA}{May 2019 - Ago 2019}
			\item Aumenté la rentabilidad de Probo entregando nuevas características como construcciones de ramas y soporte para servidores Git auto-alojados que ampliaron los casos de uso y el mercado de Probo.
			\item Reestructuré parte de la arquitectura de Probo (Node.js, Kafka, Docker API) y refactoricé servicios para cumplir con los estándares de codificación y aumentar la mantenibilidad.
		\end{rSubsection}
		
		\begin{rSubsection}{Pasante de Ingeniería de Software}{Keriton}{Filadelfia, PA}{Jun 2018 - Ago 2018}
			\item Descubrí y reporté una vulnerabilidad crítica en la que una sola credencial de usuario daba al atacante acceso y control de los datos de todos los demás usuarios.
			\item Expandí el soporte al cliente diseñando y desarrollando la arquitectura para la internacionalización en las aplicaciones cliente (AngularJS) y servidor (Node.js \& PostgreSQL).
			\item Mejoré la eficiencia y reduje el tiempo dedicado a pruebas manuales desarrollando pruebas automatizadas de extremo a extremo, usando Karma y Jasmine, en todas las aplicaciones cliente y servidor.
			\item Eliminé toda las entradas manuales de datos de pacientes procesando datos directamente de los sistemas de los hospitales a través de feeds ADT.
		\end{rSubsection}
		
	\end{rSection}
	
	%----------------------------------------------------------------------------------------
	%	SECCIÓN DE EXPERIENCIA EN CÓDIGO ABIERTO
	%----------------------------------------------------------------------------------------
	
	\begin{rSection}{OPEN SOURCE Y OTRAS EXPERIENCIAS}
		
		\begin{rSubsection}{Contribuidor}{Drupal}{}{Dic 2014 - May 2020}
			\item Nombrado uno de los 50 mejores contribuidores en 2018 y 2019 cuando tuve la oportunidad de presentar parte de mi trabajo en el principal evento anual de Drupal.
			\item Desarrollé la API Social para armonizar la autenticación con más de 25 proveedores externos (como Facebook, Instagram y Google).
			\item Aceleré el proceso de desarrollo de módulos de autenticación proporcionando una base de código común y API.
			\item Construí y mantuve más de 20 módulos que contabilizaron más de 1,000,000 descargas.
		\end{rSubsection}
		
		\hfil
		
		\begin{rSubsection}{Administrador de Org. y Mentor}{Google Code-In}{}{Nov 2015 - May 2020}
			\item Introduje a más de 100 estudiantes preuniversitarios cada año al desarrollo de Código Abierto, involucrando a muchos de ellos para continuar contribuyendo al Código Abierto después de GCI.
			\item Revisé documentación y código de más de 500 instancias de tareas cada año.
			\item Creé más de 200 tareas y coordiné actividades así como responsabilidades con más de 10 mentores cada año.
		\end{rSubsection}
		
		
		\begin{rSubsection}{Administrador de Org. y Mentor}{Google Summer of Code}{}{Mar 2017 - May 2020}
			\item Revisé más de 30 propuestas de proyectos cada año y seleccioné estudiantes basándome en la calidad de las propuestas y contribuciones previas al Código Abierto y Drupal.
			\item Mentoré a estudiantes involucrados en un proyecto de 3 meses, incluyendo revisión de código, pruebas y depuración.
		\end{rSubsection}
		
	\end{rSection}
	
	%----------------------------------------------------------------------------------------
	%	SECCIÓN DE EDUCACIÓN
	%----------------------------------------------------------------------------------------
	
	\begin{rSection}{Educación}
		
		{\bf Swarthmore College}, Swarthmore, PA \hfill {Mayo 2021} \\ 
		Licenciatura en Ciencias de la Computación y Ciencias Políticas
		
	\end{rSection}
	
	%----------------------------------------------------------------------------------------
	%	SECCIÓN DE HABILIDADES
	%----------------------------------------------------------------------------------------
	
	\begin{rSection}{Habilidades}
		\underline{Habilidades de Programación:} Python, Go, JavaScript, PHP, C, C++, CSS
		
		\underline{Tecnologías:} Flask, AWS, GCP, Drupal, Node.js, Git, Linux, Docker, Postgres, MongoDB, Kafka
	\end{rSection}
	
\end{document}